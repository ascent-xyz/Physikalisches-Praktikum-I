%%%%%%%%%%%%%%%%%%%%%%%%%%%%%%
% GENERAL

\newcommand{\e}{\mathrm{e}} % Der hier von \eul redefined
\newcommand{\imag}{\mathrm{i}}
%\newcommand{\dif}[1]{\mathrm{d}#1} not needed (fixdif package)
\newcommand{\Le}{\mleft}
\newcommand{\Ri}{\mright}
\newcommand{\RR}{\mathbb{R}}
\newcommand{\ensp}{\enspace}
\newcommand{\sbst}{\subset}
\newcommand{\id}{\mathbb{1}}
\renewcommand{\implies}{\Rightarrow}
\newcommand{\openint}[2]{{]{#1,#2}[}}
\newcommand{\diag}[1]{\mathrm{diag}\Le\{#1\Ri\}}
\DeclarePairedDelimiterXPP\Exp[1]{\operatorname{exp}}{(}{)}{}{#1} %exp Operator mit passendem Klammern Spacing oder so
\newcommand{\bv}[1]{\symbfit{#1}} % bolt vector

%%%%%%%%%%%%%%%%%%%%%%%%%%%%%%
% SYMBOLS

\newcommand{\veps}{\varepsilon}
\newcommand{\vphi}{\varphi}

%%%%%%%%%%%%%%%%%%%%%%%%%%%%%%
% TOPOLGY

\DeclareMathOperator{\accumulated}{acc}
\DeclareMathOperator{\isolated}{iso}
\DeclareMathOperator{\interior}{int}
\DeclareMathOperator{\exterior}{ext}
\newcommand{\edge}{\partial}
\DeclareMathOperator{\CauchySeq}{\text{CF}}

%%%%%%%%%%%%%%%%%%%%%%%%%%%%%%
% LINEAR ALGEBRA

\DeclareMathOperator{\trace}{Tr}
\DeclareMathOperator*{\Vektor}{\scalerel*{V}{\sum}}
\DeclareMathOperator{\core}{ker}
%\DeclareMathOperator{\span}{span}

%%%%%%%%%%%%%%%%%%%%%%%%%%%%%%
% VECTOR CALCULUS

\newcommand{\nbl}{\symbfup{\nabla}}
\DeclareMathOperator{\grad}{grad}
\DeclareMathOperator{\divgc}{div}
\DeclareMathOperator{\rot}{rot}

%%%%%%%%%%%%%%%%%%%%%%%%%%%%%%
% PHYSICS
\newcommand{\poiss}[2]{\Le\{#1,#2\Ri\}} % Poissant brackets

%\prime
%\dprime
%\trprime