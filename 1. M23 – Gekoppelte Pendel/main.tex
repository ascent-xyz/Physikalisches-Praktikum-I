
\documentclass[
12pt,
a4paper,
bibliography=totocnumbered, %Literaturverzereichnis als Eintrag ins Inhaltsverzeichnis
twoside, %Zweiseitiger Druck
BCOR=1cm, %Platz zum Lochen
%oneside, %Einseitiger Druck
]{scrartcl}

\usepackage{geometry}



\usepackage[ngerman]{babel}
\usepackage[babel=true,german=quotes]{csquotes}

\usepackage{microtype}
\usepackage{mleftright}
%\newcommand{\Le}{\mleft}
%\newcommand{\Ri}{\mright}
\usepackage[no-script, no-scriptscript, no-inner, no-close]{innerscript}

% Mathepakete (unicode-math ersetzt amssymb, amsfonts etc.)
\usepackage{amsmath,amsthm}
\usepackage{mathtools}
\usepackage{fixdif,derivative}
\usepackage{physics2}
\usephysicsmodule{ab.legacy}

\PassOptionsToPackage{math-style=ISO}{unicode-math}
\usepackage[default]{fontsetup}

\usepackage{siunitx}
\sisetup{detect-weight=true, detect-family=true,locale=DE,range-phrase={\,bis\,},list-final-separator ={\,\linebreak[0] \text{und}\,},separate-uncertainty=true,per-mode = symbol-or-fraction}
%\SI[per-mode = fraction]{1}{\meter\per\second} erzwingt auch im Fließtext die Bruchdarstellung.
\DeclareSIUnit\curie{Ci}%Zusätzliche Einheit definieren

\usepackage{tabularx, booktabs, multirow}
\usepackage{array}
\usepackage{enumitem}
\usepackage{float}
\usepackage{graphicx}
\usepackage{xurl}

\usepackage[final]{pdfpages}
\usepackage{framed, color} %Framed: Paket, mittels dessen ein Rahmen um einen Bereich definiert werden kann. Color: Lässt Farbdarstellung in Schrift, Hintergrund etc. zu

\usepackage{scrlayer-scrpage} %Header für die KOMA-script-Klasse

%\usepackage[square,numbers]{natbib}
\usepackage{subfigure} %Mehrere Bilder in einer Figure-Umgebung

\usepackage[bookmarks,colorlinks=true]{hyperref}
\hypersetup{
	colorlinks,
	linktocpage,
	citecolor=black,
	filecolor=black,
	linkcolor=black,
	urlcolor=black,
	pdftitle=X,
	pdfauthor=JM-VS
}

\usepackage[backend=biber, style=chem-angew]{biblatex}
\addbibresource{lit.bib}

\numberwithin{equation}{section} % Die Nummerierung von Gleichungen bekommt die jeweilige Section-Nummer als Präfix

\setlength{\parindent}{0pt} %Einrücktiefe von neuen Absätzen
\setlength{\parskip}{6pt} %Abstand von Absätzen

\pagestyle{scrheadings}%Kopf und Fußzeilen
\ohead{\textbf{\GRUPPENNR\ - \VERSUCHSNR}} %Header oben links auf linker Seite (ungerade Seitenzahl) und oben rechts auf rechter Seite (gerade Seitenzahl), beinhaltet gruppennummer und Versuchskürzel. Im Fall eine einseitigen Dokuments: Header oben rechts
\ihead{\VerfasserEINS\;\&\;\VerfasserZWEI} %Header oben rechts auf linker Seite und oben links auf rechter Seite. Beinhaltet die Namen der Verfasser. Im Fall eine einseitigen Dokuments: Header oben links!
\ofoot{\thepage} %Footer unten links auf linker und unten rechts auf rechter Seite, enthält die jeweilige Seitenzahl. Im Fall eines einseitigen Elements: Footer unten rechts!
\cfoot{\empty} %Mittig unten im Footer soll nichts eingetragen werden
\ifoot{\empty} %Footer unten rechts auf linker und unten links auf rechter Seite. Hier ebenfalls leer.

\newcommand{\tz}{T_{\text{II}}}
\newcommand{\ts}{T_{\text{S}}}
\newcommand{\tgl}{T_{\text{gl}}}
\newcommand{\tgeg}{T_{\text{geg}}}
\newcommand{\omz}{\omega_{\text{II}}}
\newcommand{\omgl}{\omega_{\text{gl}}}
\newcommand{\omgeg}{\omega_{\text{geg}}}
\newcommand{\kB}{k_{\text{B}}}

% Hier können die individuellen Anpassungen vorgenommen werden, die sich auf das Titelblatt und die Kopfzeilen auswirken.

\newcommand{\VERSUCHSDATUM}{24.09.2025}
\newcommand{\PROTOKOLLDATUM}{\today}

\newcommand{\VerfasserEINS}{Julian Molt}
\newcommand{\MatNoEINS}{3803097}
\newcommand{\StudiengangEINS}{Physik}

\newcommand{\VerfasserZWEI}{Valentin Stopper}
\newcommand{\MatNoZWEI}{3774391}
\newcommand{\StudiengangZWEI}{Physik}

\newcommand{\BETREUER}{Lara Zaiser}
\newcommand{\GRUPPENNR}{A-016}

\newcommand{\VERSUCHSNR}{M23}
\newcommand{\VERSUCHSNAME}{Gekoppelte Pendel}

\newcommand{\lh}{\ell_{\mathrm{H}}}
\newcommand{\ls}{\ell_{\mathrm{S}}}


\begin{document}

\thispagestyle{empty}

\begin{titlepage}

	\begin{center}
		\Huge{\textbf{\VERSUCHSNR\ -- \VERSUCHSNAME}}\\
		\vspace{10mm}
		\Large{Protokoll zum Versuch des Physikalischen Praktikums I von \\ \textbf{\VerfasserEINS\;\& \VerfasserZWEI}}\\
		\vspace{10mm}
		\Large{Universität Stuttgart}\\
	\end{center}
	\vspace{1cm}
	\begin{center}
		\begin{tabular}{ll}
			\large{Verfasser:}		& \large{\VerfasserEINS\;(\StudiengangEINS),} \\
			& \large{\MatNoEINS} \\
			\vspace{0cm}\\
			& \large{\VerfasserZWEI\;(\StudiengangZWEI),} \\
			& \large{\MatNoZWEI} \\
			\vspace{0cm}\\
			\large{Gruppennummer:}	& \large{\GRUPPENNR} \\
			\vspace{0cm}\\
			\large{Versuchsdatum:}	& \large{\VERSUCHSDATUM} \\
			\vspace{0cm}\\
			\large{Betreuerin:}		& \large{\BETREUER}
		\end{tabular}
	\end{center}
	\vspace{15mm}

	\begin{center}
		Stuttgart, den \PROTOKOLLDATUM
	\end{center}

\end{titlepage}

\thispagestyle{empty}

\tableofcontents

\clearpage %Neue Seite, davor werden alle noch ausstehenden Grafiken/Tabellen platziert.

\renewcommand{\thepage}{\arabic{page}}
\setcounter{page}{1}


% Die erste eckige Klammer ist optional, die darin angegebene Bezeichnung steht im Inhaltsverzeichnis anstelle des hinteren (längeren) Namens.
\section[Versuchsziel]{Versuchsziel und Versuchsmethode}



\section{Grundlagen}




\section[Messprinzip]{Messprinzip mit Skizze und Versuchsablauf}

\begin{figure}[H]
	\centering{\includegraphics[width=0.4\textwidth]{Aufbau}}
	\caption{Aufbau der Pendel, die hier bei \(\lh = \qty{70}{\centi\meter}\) gekoppelt sind.}
	\label{fig:aufbau}
\end{figure}



\section[Formeln]{Formeln}

\section{Messwerte}

\begin{table}[H]
	\begin{tabular*}{\textwidth}{@{\extracolsep{\fill}}@{\hspace{5pt}}lrr@{\hspace{5pt}}}
		\toprule
		Parameter & Pendel \num{1} & Pendel \num{2}\\
		\midrule
		Masse des Pendelkörpers \(m\)\,/\,\si{\kilogram} & \num{176,97e-3}   & \num{174,95e-3}\\
		Masse des Hakens \(m_{\text{H}}\)\,/\,\si{\kilogram} & \num{16,06e-3}   & \num{15,80e-3}\\
		Masse der Stange & \num{131,40e-3} & \num{131,27e-3}\\
		\(L\)\,/\, \si{\kilogram} & \num{0,804} & \num{0,800}\\
		\(\lh\) \,/\, \si{\meter} & \num{0,4} & \num{0,4}\\
		\(L_{\text{St}}\)\,/\, \si{\meter} & \num{0,87} & \num{0,87}\\
		\bottomrule
	\end{tabular*}
	\caption{Ungekoppelter gleichsinniger Fall \label{tbl:dimensions}}
\end{table}

\subsection{\texorpdfstring{\qty{40}{\centi\meter}}{40 cm}}

\begin{table}[H]
	\begin{tabular*}{\textwidth}{@{\extracolsep{\fill}}@{\hspace{5pt}}lrr@{\hspace{5pt}}}
		\toprule
		Pendel & \(t_0\)\,/\,\(\si{\second}\) & \(t_1\)\,/\,\(\si{\second}\)\\
		\midrule
		1 & \num{1,4}   & \num{35,5}\\
		2 & \num{1,3}   & \num{35,4}\\
		\bottomrule
	\end{tabular*}
	\caption{Ungekoppelter gleichsinniger Fall \label{tbl:ngekgl40}}
\end{table}

\begin{table}[H]
	\begin{tabular*}{\textwidth}{@{\extracolsep{\fill}}@{\hspace{5pt}}lrr@{\hspace{5pt}}}
		\toprule
		Pendel & \(t_0\)\,/\,\(\si{\second}\) & \(t_1\)\,/\,\(\si{\second}\)\\
		\midrule
		1 & \num{0,9}   & \num{34,8}\\
		2 & \num{0,8}   & \num{34,8}\\
		\bottomrule
	\end{tabular*}
	\caption{Gekoppelter gleichsinniger Fall \label{tbl:gekgl40}}
\end{table}

\begin{table}[H]
	\begin{tabular*}{\textwidth}{@{\extracolsep{\fill}}@{\hspace{5pt}}lrr@{\hspace{5pt}}}
		\toprule
		Pendel & \(t_0\)\,/\,\(\si{\second}\) & \(t_1\)\,/\,\(\si{\second}\)\\
		\midrule
		1 & \num{0,4}   & \num{33,3}\\
		2 & \num{1,2}   & \num{34,1}\\
		\bottomrule
	\end{tabular*}
	\caption{Gekoppelter gegensinniger Fall \label{tbl:gekgeg40}}
\end{table}

\begin{table}[H]
	\begin{tabular*}{\textwidth}{@{\extracolsep{\fill}}@{\hspace{5pt}}lrrrrr@{\hspace{5pt}}}
		\toprule
		Pendel & \(t_0\)\,/\,\(\si{\second}\) & \(t_1\)\,/\,\(\si{\second}\)& \(t_2\)\,/\,\(\si{\second}\)& \(t_3\)\,/\,\(\si{\second}\)& \(t_4\)\,/\,\(\si{\second}\)\\
		\midrule
		1 & \num{0,0}   & \num{52,4} & \num{105,8} & \num{160,0} & \num{211,6}\\
		2 & \num{25,4}   & \num{79,5} & \num{132,9} & \num{185,3} & \num{237,8}\\
		\bottomrule
	\end{tabular*}
	\caption{Schwebungsfall \label{tbl:schweb40}}
\end{table}

\subsection{\texorpdfstring{\qty{55}{\centi\meter}}{55 cm}}

\begin{table}[H]
	\begin{tabular*}{\textwidth}{@{\extracolsep{\fill}}@{\hspace{5pt}}lrr@{\hspace{5pt}}}
		\toprule
		Pendel & \(t_0\)\,/\,\(\si{\second}\) & \(t_1\)\,/\,\(\si{\second}\)\\
		\midrule
		1 & \num{0,8}   & \num{35,0}\\
		2 & \num{0,8}   & \num{34,8}\\
		\bottomrule
	\end{tabular*}
	\caption{Ungekoppelter gleichsinniger Fall \label{tbl:ngekgl55}}
\end{table}

\begin{table}[H]
	\begin{tabular*}{\textwidth}{@{\extracolsep{\fill}}@{\hspace{5pt}}lrr@{\hspace{5pt}}}
		\toprule
		Pendel & \(t_0\)\,/\,\(\si{\second}\) & \(t_1\)\,/\,\(\si{\second}\)\\
		\midrule
		1 & \num{0,5}   & \num{34,5}\\
		2 & \num{0,5}   & \num{34,5}\\
		\bottomrule
	\end{tabular*}
	\caption{Gekoppelter gleichsinniger Fall \label{tbl:gekgl55}}
\end{table}

\begin{table}[H]
	\begin{tabular*}{\textwidth}{@{\extracolsep{\fill}}@{\hspace{5pt}}lrr@{\hspace{5pt}}}
		\toprule
		Pendel & \(t_0\)\,/\,\(\si{\second}\) & \(t_1\)\,/\,\(\si{\second}\)\\
		\midrule
		1 & \num{1,4}   & \num{33,6}\\
		2 & \num{0,6}   & \num{32,9}\\
		\bottomrule
	\end{tabular*}
	\caption{Gekoppelter gegensinniger Fall \label{tbl:gekgeg55}}
\end{table}

\begin{table}[H]
	\begin{tabular*}{\textwidth}{@{\extracolsep{\fill}}@{\hspace{5pt}}lrrrrr@{\hspace{5pt}}}
		\toprule
		Pendel & \(t_0\)\,/\,\(\si{\second}\) & \(t_1\)\,/\,\(\si{\second}\)& \(t_2\)\,/\,\(\si{\second}\)& \(t_3\)\,/\,\(\si{\second}\)& \(t_4\)\,/\,\(\si{\second}\)\\
		\midrule
		1 & \num{14,8}   & \num{44,9} & \num{74,6} & \num{105,2} & \num{134,8}\\
		2 & \num{0,0}   & \num{29,5} & \num{60,2} & \num{89,8} & \num{119,6}\\
		\bottomrule
	\end{tabular*}
	\caption{Schwebungsfall \label{tbl:schweb55}}
\end{table}

\subsection{\texorpdfstring{\qty{70}{\centi\meter}}{70 cm}}

\begin{table}[H]
	\begin{tabular*}{\textwidth}{@{\extracolsep{\fill}}@{\hspace{5pt}}lrr@{\hspace{5pt}}}
		\toprule
		Pendel & \(t_0\)\,/\,\(\si{\second}\) & \(t_1\)\,/\,\(\si{\second}\)\\
		\midrule
		1 & \num{1,5}   & \num{35,8}\\
		2 & \num{1,5}   & \num{35,7}\\
		\bottomrule
	\end{tabular*}
	\caption{Ungekoppelter gleichsinniger Fall \label{tbl:ngekgl70}}
\end{table}

\begin{table}[H]
	\begin{tabular*}{\textwidth}{@{\extracolsep{\fill}}@{\hspace{5pt}}lrr@{\hspace{5pt}}}
		\toprule
		Pendel & \(t_0\)\,/\,\(\si{\second}\) & \(t_1\)\,/\,\(\si{\second}\)\\
		\midrule
		1 & \num{0,5}   & \num{34,7}\\
		2 & \num{0,5}   & \num{34,7}\\
		\bottomrule
	\end{tabular*}
	\caption{Gekoppelter gleichsinniger Fall \label{tbl:gekgl70}}
\end{table}

\begin{table}[H]
	\begin{tabular*}{\textwidth}{@{\extracolsep{\fill}}@{\hspace{5pt}}lrr@{\hspace{5pt}}}
		\toprule
		Pendel & \(t_0\)\,/\,\(\si{\second}\) & \(t_1\)\,/\,\(\si{\second}\)\\
		\midrule
		1 & \num{1,5}   & \num{33,0}\\
		2 & \num{0,7}   & \num{32,2}\\
		\bottomrule
	\end{tabular*}
	\caption{Gekoppelter gegensinniger Fall \label{tbl:gekgeg70}}
\end{table}

\begin{table}[H]
	\begin{tabular*}{\textwidth}{@{\extracolsep{\fill}}@{\hspace{5pt}}lrrrrr@{\hspace{5pt}}}
		\toprule
		Pendel & \(t_0\)\,/\,\(\si{\second}\) & \(t_1\)\,/\,\(\si{\second}\)& \(t_2\)\,/\,\(\si{\second}\)& \(t_3\)\,/\,\(\si{\second}\)& \(t_4\)\,/\,\(\si{\second}\)\\
		\midrule
		1 & \num{0,0}   & \num{18,1} & \num{38,7} & \num{58,3} & \num{77,8}\\
		2 & \num{9,0}   & \num{29,5} & \num{49,0} & \num{68,6} & \num{88,3}\\
		\bottomrule
	\end{tabular*}
	\caption{Schwebungsfall \label{tbl:schweb70}}
\end{table}

\begin{table}[H]
	\begin{tabular*}{\textwidth}{@{\extracolsep{\fill}}@{\hspace{5pt}}lrrrrr@{\hspace{5pt}}}
		\toprule
		Pendel & \(t_0\)\,/\,\(\si{\second}\) & \(t_1\)\,/\,\(\si{\second}\)& \(t_2\)\,/\,\(\si{\second}\)& \(t_3\)\,/\,\(\si{\second}\)& \(t_4\)\,/\,\(\si{\second}\)\\
		\midrule
		1 & \num{6,2}   & \num{26,7} & \num{47,1} & \num{66,9} & \num{85,6}\\
		2 & \num{16,4}   & \num{36,9} & \num{57,3} & \num{77,1} & \num{97,4}\\
		\bottomrule
	\end{tabular*}
	\caption{Schwebungsfall für unterschiedlich ausgelenkte Massen \label{tbl:schwebX70}}
\end{table}

\section{Auswertung}

\(T_0\) berechnet sich aus den Daten, die in \autoref{tbl:ngekgl40} eingetragen sind durch
\begin{equation}
	T_0 = \frac{t_1 - t_0}{20} \,,
\end{equation}
da zwischen \(t_0\) und \(t_1\) \num{20} Perioden durchlaufen wurden.
Damit ergibt sich für Pendel \num{1}
\begin{equation}
	T_{0,1} = \frac{\qty{35,5}{\second} - \qty{1,4}{\second}}{20} = \qty{1,7}{\second}
\end{equation}
und für Pendel \num{2} analog  ebenfalls \(T_{0,2} =\qty{1,7}{\second}\).

Für den Kopplungsgrad mit \(\lh = \qty{40}{\centi\meter}\) ergeben sich folgende Periodendauern.

\begin{table}[H]
	\begin{tabular*}{\textwidth}{@{\extracolsep{\fill}}@{\hspace{5pt}}lrrr@{\hspace{5pt}}}
		\toprule
		Periodendauer & Links & Rechts & Mittel\\
		\midrule
		\(T_{\text{gl}}\) & \qty{1,7}{\second} & \qty{1,7}{\second} & \qty{1,7}{\second}\\
		\(T_{\text{geg}}\) & \qty{1,6}{\second} & \qty{1,6}{\second} & \qty{1,6}{\second}\\
		\(T_{\text{II}}\) & \qty{1,6}{\second} & \qty{1,6}{\second} & \qty{1,6}{\second}\\
		\(T_{\text{S}}\) & \qty{52,9}{\second} & \qty{53,1}{\second} & \qty{53,0}{\second} \\
		\bottomrule
	\end{tabular*}
	\caption{Periodendauern für \(\lh\) \label{tbl:res40}}
\end{table}

Für den Kopplungsgrad mit \(\lh = \qty{55}{\centi\meter}\) ergeben sich folgende Periodendauern.

\begin{table}[H]
	\begin{tabular*}{\textwidth}{@{\extracolsep{\fill}}@{\hspace{5pt}}lrrr@{\hspace{5pt}}}
		\toprule
		Periodendauer & Links & Rechts & Mittel\\
		\midrule
		\(T_{\text{gl}}\) & \qty{1,7}{\second} & \qty{1,7}{\second} & \qty{1,7}{\second}\\
		\(T_{\text{geg}}\) & \qty{1,6}{\second} & \qty{1,6}{\second} & \qty{1,6}{\second}\\
		\(T_{\text{II}}\) & \qty{1,6}{\second} & \qty{1,6}{\second} & \qty{1,6}{\second}\\
		\(T_{\text{S}}\) & \qty{30,0}{\second} & \qty{29,9}{\second} & \qty{30,0}{\second} \\
		\bottomrule
	\end{tabular*}
	\caption{Periodendauern für \(\lh\) \label{tbl:res55}}
\end{table}

Für den Kopplungsgrad mit \(\lh = \qty{70}{\centi\meter}\) ergeben sich folgende Periodendauern.

\begin{table}[H]
	\begin{tabular*}{\textwidth}{@{\extracolsep{\fill}}@{\hspace{5pt}}lrrr@{\hspace{5pt}}}
		\toprule
		Periodendauer & Links & Rechts & Mittel\\
		\midrule
		\(T_{\text{gl}}\) & \qty{1,7}{\second} & \qty{1,7}{\second} & \qty{1,7}{\second}\\
		\(T_{\text{geg}}\) & \qty{1,6}{\second} & \qty{1,6}{\second} & \qty{1,6}{\second}\\
		\(T_{\text{II}}\) & \qty{1,6}{\second} & \qty{1,6}{\second} & \qty{1,6}{\second}\\
		\(T_{\text{S}}\) & \qty{19,5}{\second} & \qty{19,8}{\second} & \qty{19,6}{\second} \\
		\bottomrule
	\end{tabular*}
	\caption{Periodendauern für \(\lh\) \label{tbl:res70}}
\end{table}

\begin{figure}[H]
	\centering{\includegraphics[width=0.9\textwidth]{SCHWEBUNG_70cm.pdf}}
	\caption{Exemplarisches \(x(t)\)-Diagramm des Schwebungsfalles bei \(\lh = \qty{70}{\centi\meter}\)}
	\label{fig:schweb}
\end{figure}

\begin{figure}[H]
	\centering{\includegraphics[width=0.9\textwidth]{gleichsinnige_FUNDAMENTALSCHWINGUNG_70cm.pdf}}
	\caption{Exemplarisches \(x(t)\)-Diagramm der gleichsinnigen Fundamentalschwingung bei \(\lh = \qty{70}{\centi\meter}\)}
	\label{fig:gl70}
\end{figure}

\begin{figure}[H]
	\centering{\includegraphics[width=0.9\textwidth]{gegensinnige_FUNDAMENTALSCHWINGUNG_70cm.pdf}}
	\caption{Exemplarisches \(x(t)\)-Diagramm der gegensinnigen Fundamentalschwingung bei \(\lh = \qty{70}{\centi\meter}\)}
	\label{fig:geg70}
\end{figure}


%\includepdf[pages=-]{SCHWEBUNG_70cm.pdf}
%\includepdf[pages=-]{gleichsinnige_FUNDAMENTALSCHWINGUNG_70cm.pdf}
%\includepdf[pages=-]{gegensinnige_FUNDAMENTALSCHWINGUNG_70cm.pdf}

Nach %\autoref{eq:Drehmoment}
wird das Gesamtträgheitsmoment mit den Daten aus \autoref{tbl:dimensions} berechnet. Für das erste Pendel ergibt sich das Trägheitsmoment
\begin{equation}
	\begin{split}
		J_1 = \frac{1}{3} \cdot \qty{131,40e-3}{\kilogram} \cdot (\qty{0,87}{\meter})^2 &+ \qty{16,06e-3}{\kilogram} \cdot (\qty{0,4}{\meter})^2
		\\&+ \qty{176,97e-3}{\kilogram} \cdot (\qty{0,804}{\meter})^2 = \qty{0,150}{\kilogram\meter\squared}
	\end{split}
\end{equation}
und für das zweite Pendel \(J_2 = \qty{0,148}{\kilogram\meter\squared}\).

Die Eigenfrequenz \(\omega_0\) wird mittels %\autoref{eq:eig}
berechnet. Dafür muss zuerst die Lage des Schwerpunktes \(\ls\) mit %\autoref{eq:schwerp}
berechnet werden und ergibt
\begin{equation}
	\ls = \frac{Lm + \lh m_{\text{H}} + \tfrac{1}{2} L_{\text{St}} m_{\text{St}}}{m + m_{\text{H}} + m_{\text{St}}}
\end{equation}
\begin{equation}
	\begin{split}
		\ell_{\text{S},1} &= \frac{\qty{0,804}{\meter} \cdot \qty{176,97e-3}{\kilogram} + \qty{0,4}{\meter} \cdot \qty{16,06e-3}{\kilogram} + \tfrac{1}{2}\qty{0,87}{\meter} \cdot \qty{131,40e-3}{\kilogram}}{\mleft(\num{176,97e-3} + \num{16,06e-3} + \num{131,40e-3}\mright)\si{\kilogram}}\\
		&= \qty{0,635}{\meter}
	\end{split}
\end{equation}
Für \(\ell_{\text{S},2}\) ergibt sich \(\ell_{\text{S},1} = \qty{0,632}{m}\).

\section{Fehlerrechnung}

\section{Zusammenfassung}


\begin{thebibliography}{999}
	\bibitem{Quelle} Versuchsanleitung zu (Abgerufen am 1.04.2050)
\end{thebibliography}


\section{Anhang}

\includepdf[pages=-]{Messprotokoll.pdf}

\end{document}