\usepackage[default]{fontsetup}

\usepackage[ngerman]{babel}
\usepackage[babel=true,german=quotes]{csquotes}

%\usepackage[onehalfspacing]{setspace}

% Mathepakete (unicode-math ersetzt amssymb, amsfonts etc.)
\usepackage{amsmath,amsthm}
\usepackage{microtype}
%\usepackage[no-script, no-inner]{innerscript}
\usepackage[no-script, no-inner, no-close]{innerscript}
\usepackage{mathtools}
\usepackage{mleftright}
\usepackage{fixdif,derivative}
\usepackage{physics2}
\usephysicsmodule{ab.legacy}

\usepackage{unicode-math}
\setmathfont{NewCMMath-Book}
\setmathfont{NewCMMath-Book}[range=cal, StylisticSet=1]
\setmathfont{NewCMMath-Book}[range=scr]

\usepackage{siunitx}
\sisetup{detect-weight=true, detect-family=true,locale=DE,range-phrase={\,bis\,},list-final-separator ={\,\linebreak[0] \text{und}\,},separate-uncertainty=true,per-mode = symbol-or-fraction}
%\SI[per-mode = fraction]{1}{\meter\per\second} erzwingt auch im Fließtext die Bruchdarstellung.
\DeclareSIUnit\curie{Ci}%Zusätzliche Einheit definieren

\usepackage{enumitem}
\usepackage{array}
\usepackage{graphicx}
\usepackage{geometry}
\usepackage{tabularx, booktabs, multirow}
\usepackage{xurl}
\usepackage{float}
\usepackage[final]{pdfpages}
\usepackage{framed, color} %Framed: Paket, mittels dessen ein Rahmen um einen Bereich definiert werden kann. Color: Lässt Farbdarstellung in Schrift, Hintergrund etc. zu
\usepackage{scrlayer-scrpage} %Header für die KOMA-script-Klasse

\usepackage[square,numbers]{natbib}
\usepackage{subfigure} %Mehrere Bilder in einer Figure-Umgebung

\usepackage[bookmarks,colorlinks=true]{hyperref}
\hypersetup{
	colorlinks,
	linktocpage,
	citecolor=black,
	filecolor=black,
	linkcolor=black,
	urlcolor=black
}

\numberwithin{equation}{section} % Die Nummerierung von Gleichungen bekommt die jeweilige Section-Nummer als Präfix

%\setlength{\parindent}{0 mm} %Einrücktiefe von neuen Absätzen
%\setlength{\parskip}{2 mm} %Abstand von Absätzen

\pagestyle{scrheadings}%Kopf und Fußzeilen
\ohead{\textbf{\GRUPPENNR\ - \VERSUCHSNR}} %Header oben links auf linker Seite (ungerade Seitenzahl) und oben rechts auf rechter Seite (gerade Seitenzahl), beinhaltet gruppennummer und Versuchskürzel. Im Fall eine einseitigen Dokuments: Header oben rechts
\ihead{\VerfasserEINS\;\&\;\VerfasserZWEI} %Header oben rechts auf linker Seite und oben links auf rechter Seite. Beinhaltet die Namen der Verfasser. Im Fall eine einseitigen Dokuments: Header oben links!
\ofoot{\thepage} %Footer unten links auf linker und unten rechts auf rechter Seite, enthält die jeweilige Seitenzahl. Im Fall eines einseitigen Elements: Footer unten rechts!
\cfoot{\empty} %Mittig unten im Footer soll nichts eingetragen werden
\ifoot{\empty} %Footer unten rechts auf linker und unten links auf rechter Seite. Hier ebenfalls leer.