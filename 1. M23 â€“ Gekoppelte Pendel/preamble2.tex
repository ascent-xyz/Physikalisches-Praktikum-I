\documentclass[12pt,a4paper]{scrartcl}

\usepackage[lmargin=2.5cm, rmargin=2.5cm, top=100pt, bottom=100pt]{geometry}
%\geometry{bottom=100pt} \geometry{top=100pt} %Definiert den oberen und unteren Rand
\usepackage[onehalfspacing]{setspace} %Bestimmt den Zeilenabstand

\usepackage[default]{fontsetup}
%\usepackage{fontspec} %mit lmodern
%\setmainfont{Latin Modern Roman}

\usepackage[ngerman]{babel}
\usepackage[babel=true,german=quotes]{csquotes}

\usepackage{microtype}

\usepackage{mleftright}

\usepackage{amsmath}
\usepackage{mathtools}
\usepackage{mleftright}
\usepackage{fixdif,derivative}
\usepackage{physics2}
\usephysicsmodule{ab.legacy}

\usepackage{unicode-math}
\setmathfont[StylisticSet=1]{NewCMMath-Book}
%\setmathfont{Latin Modern Math}
%\setmathfont{Latin Modern Math}[range=\hbar]

\usepackage{siunitx} %Einheiten
\sisetup{locale = DE}

\usepackage{tabularx, booktabs, multirow} %Für bessere Tabellen und einfachen Excel to LaTeX import
\usepackage{array} %Bessere Formatierung / Umgebung für mehrzeilige Formeln
\usepackage{float} %Großes H zum fixieren von Abbildungen/Tabellen
\usepackage{graphicx}
\usepackage{epstopdf} %Um .eps-Bilder in LaTeX verwenden zu können

%\usepackage{svg}
%\svgsetup{inkscapelatex=true}

%\usepackage{upgreek} %Für nicht kursive griechische Buchstaben

\usepackage{chemformula}% mächtiges Chemiepaket(Rkt. Gl., Oxidationszahlen etc.)
\usepackage{chemgreek}% für chemformula
%\usepackage{chemmacros}% für chemformula
\usepackage{tikz} %Vektorgrafziken im LaTeX-eigenen Format, z. B. zum Export aus QtiPlot

\usepackage{xurl}
\usepackage{hyperref}
\hypersetup{
	colorlinks=true,
	linkcolor=black,
	urlcolor=blue,
	citecolor=blue,
	pdftitle=Versuch1,
	pdfauthor=JM-VS
}

\usepackage[font=small]{caption}

\usepackage[backend=biber, style=chem-angew]{biblatex} %Literaturverzeichnis
\addbibresource{lit.bib} %Literaturverzeichnis, zu finden als Datei lit.bib im selben Ordner wie diese .tex-Datei
\captionsetup{format=plain}

%\parindent0pt %Kein Einzug am Anfang von Absätzen
%\sloppy %Besserer Blocksatz, naja..

%\newcommand{\celsius}{^{\circ}\mathrm{C}} %Ermöglicht es, \celsius für die Einheit °C zu verwenden UNICODE!!!
%\renewcaptionname{ngerman}{\figurename}{Abb.} %Umbenennung Abbildungen, optional
%\renewcaptionname{ngerman}{\tablename}{Tab.} %Umbenennung Tabellen, optional

%%%%%%%%%%%%%%%%%%%%%%%%%%%%%%%
% GENERAL

\newcommand{\e}{\mathrm{e}} % Der hier von \eul redefined
\newcommand{\imag}{\mathrm{i}}
%\newcommand{\dif}[1]{\mathrm{d}#1} not needed (fixdif package)
\newcommand{\Le}{\mleft}
\newcommand{\Ri}{\mright}
\newcommand{\RR}{\mathbb{R}}
\newcommand{\ensp}{\enspace}
\newcommand{\sbst}{\subset}
\newcommand{\id}{\mathbb{1}}
\renewcommand{\implies}{\Rightarrow}
\newcommand{\openint}[2]{{]{#1,#2}[}}
\newcommand{\diag}[1]{\mathrm{diag}\Le\{#1\Ri\}}
\DeclarePairedDelimiterXPP\Exp[1]{\operatorname{exp}}{(}{)}{}{#1} %exp Operator mit passendem Klammern Spacing oder so
\newcommand{\bv}[1]{\symbfit{#1}} % bolt vector

%%%%%%%%%%%%%%%%%%%%%%%%%%%%%%
% SYMBOLS

\newcommand{\veps}{\varepsilon}
\newcommand{\vphi}{\varphi}

%%%%%%%%%%%%%%%%%%%%%%%%%%%%%%
% TOPOLGY

\DeclareMathOperator{\accumulated}{acc}
\DeclareMathOperator{\isolated}{iso}
\DeclareMathOperator{\interior}{int}
\DeclareMathOperator{\exterior}{ext}
\newcommand{\edge}{\partial}
\DeclareMathOperator{\CauchySeq}{\text{CF}}

%%%%%%%%%%%%%%%%%%%%%%%%%%%%%%
% LINEAR ALGEBRA

\DeclareMathOperator{\trace}{Tr}
\DeclareMathOperator*{\Vektor}{\scalerel*{V}{\sum}}
\DeclareMathOperator{\core}{ker}
%\DeclareMathOperator{\span}{span}

%%%%%%%%%%%%%%%%%%%%%%%%%%%%%%
% VECTOR CALCULUS

\newcommand{\nbl}{\symbfup{\nabla}}
\DeclareMathOperator{\grad}{grad}
\DeclareMathOperator{\divgc}{div}
\DeclareMathOperator{\rot}{rot}

%%%%%%%%%%%%%%%%%%%%%%%%%%%%%%
% PHYSICS
\newcommand{\poiss}[2]{\Le\{#1,#2\Ri\}} % Poissant brackets

%\prime
%\dprime
%\trprime
\usepackage{mleftright}
\newcommand{\Le}{\mleft}
\newcommand{\Ri}{\mright}